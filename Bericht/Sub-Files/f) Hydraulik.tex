\section{Hydraulik}
\label{sec:Hydraulik}
\subsection{f}
Aus dem auf die Kollektorfläche bezogenen Durchfluss im Kollektorkreis von 40 $\frac{l}{m^2\cdot h}$
und der zuvor berechneten Kollektorfläche ergibt sich ein Volumenstrom \.V $=793,2 \frac{l}{h}$.
Gemeinsam mit der anglegten Strömungsgeschwindigkeit $v_R = 0,7 \frac{m}{s}$ ergibt sich mit \autoref{eq:Durchmesser} \cite[S.42]{Sick22}
ein minimaler Rohrinnen-Durchmesser $d_{R,min}$ von rund 19,33 mm.

\begin{equation}
    d_{R,min}=\sqrt{\frac{4 \cdot \dot{V}}{\pi \cdot v_R}}=19,33 mm
    \label{eq:Durchmesser}
\end{equation}
\vspace{\baselineskip}
Der Durchmesser mit dem im Weiteren gearbeitet wird, ist der nächstegrößere DIN-genormte Durchmesser,
in diesem Fall DIN20 mit einem Rohrinnen-Durchmesser von 20 mm.\\