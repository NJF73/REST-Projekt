\subsection{e}
Für die Dimensionierung des Speichers wird als Faustregel angenommen, dass ca. 50 l Speichervolumen je 
$m^2$ Kollektorfläche benötigt werden \cite[S.122]{Sick22}, anhand dieser Annnahme ergibt sich ein benötigtes 
Speichervolumen von 924 Litern.\\
Passend zum gewählten Kollektor und dem Anforderungsprofil wurde der "'Viessmann Vitocell 360-M"'-Speicher mit einem Volumen von 950 Litern gewählt.\\
