\subsection{h}
Der Druckverlust über die Rohrleitung des Kollektorfeldes berechnet sich nach \autoref{eq:Verrohrung}.
Hierbei wurde die vorbestimmte Rohrleitung genutzt, welche eine neue Kupferleitung ist,
bei der die einfache Leitungslänge 12 m beträgt.\\
Zuvor muss die Reynoldszahl berechnet werden, aus der sich die benötigte Rohrreibungszahl $\lambda$ 
ergibt. Hierbei ist zu beachten, dass je nach Reynoldszahl die Formel für die Rohrreibungszahl $\lambda$
unterschiedlich ist. Das kommt zu stande, da für laminare und turbolente Strömungen unterschiedliche
Berechnungen nötig sind. In diesem Fall handelt es sich um eine turbolente Strömung.\\
\begin{equation}
    \lambda =  \frac{0,316}{Re^{0,25}} = 0,0442
\end{equation}
\begin{equation}
    \Delta p_r = \frac{1}{2} \cdot \lambda \cdot \rho \cdot v^2_R \cdot \frac{L_R}{d_R} = 11757,43 Pa
        \label{eq:Verrohrung}
\end{equation}
\vspace{\baselineskip}
\begin{center}
    Reynoldszahl $Re = \frac{v_R \cdot d_R}{\upsilon} = 2614,39$\\
    kinematische Viskosität $\upsilon =5 \cdot 10^{-6} \frac{m^2}{s}$\\
    Länge des Rohres $L_R = 24 m$
\end{center}

Das Ergebnis dieser Berechnung ist ein Druckverlust in der Rohrleitung von $\Delta p_r = 11757,43 Pa$.
Anschließend wird die Berücksichtung der Druckverluste durch die Einzelwiderstände und im Wärmeübertrager
vereinfacht und es wird angenommen, dass diese in etwa 45\% des Verlustes in der Rohrleitung betragen.
Es ergibt sich ein Verlust von $\Delta p_{EW,WÜ} = 5290,84 Pa$.