\subsection{b}
Zur korrekten Dimensionierung der Anlage ist die Betrachtung des Gesamtwärmebedarfs von zentraler Rolle.
Auf den Gesamtwärmebedarf haben vor allem die Dämm- und Speichereigenschaften des Gebäudes, sowie das Nutzungsprofil der Bewohner.\\
Für das betrachtete Gebäude werden $160 m^2$ Wohnfläche bei einem spezifischen jährlichen Heizwärmebedarf von $50 \frac{kWh}{m^2 \cdot a}$ angegeben.
Die Multiplikation dieser beiden Werte ergibt für den mittleren Heizwärmebedarf $Q_H = 8000 \frac{kWh}{a}$.\\
Um eine begründete Abschätzung der Schwankungsbreite für diesen Wert zu erhalten, wurden Daten über den Gesamtverbrauch im Bereich "'Space Heating"' in Deutschland von 2012 bis 2021 verwendet.\cite{EuroStat}\\
Mit Hilfe der Berechnung der relativen mittleren Abweichung der Werte konnte eine Toleranzgrenze von $\Delta Q_H = \pm 287,78 \frac{kWh}{a}$ ermittelt werden.\\
Anschließend wurde der Warmwasserwärmebedarf $Q_{WW}$ für mittleren, niedrigen und hohen Verbrauch \cite[S.119]{Sick22} berechnet.
Mittels \autoref{eq:Warmwasser} ergab sich für den betrachteten 4-Personen-Haushalt ein mittlerer Warmwasserwämrmebedarf inklusive Abweichung von $Q_{WW,mit} = 2711,025 \frac{kWh}{a} \pm 1355,513 \frac{kWh}{a}$.

\begin{equation}
    Q_{WW}= c_p \cdot m \cdot Bewohnerzahl \cdot \rho \cdot \Delta T
    \label{eq:Warmwasser}
\end{equation}
\vspace{\baselineskip}
\begin{center}
    spez. Wärmekapazität $c_p = 3,68 \frac{kJ}{kg \cdot K}$\\
    Dichte $\rho = 1038 \frac{kg}{m^3}$\\
    Temperaturdifferenz $\Delta T = 35 K$ 
\end{center}

Des Weiteren ist der Wärmebedarf der Zirkulation zu berechnen, da auch dieser sich auf den Gesamtbedarf auswirkt.
Hierfür wird die folgende Formel verwendet \cite[S.73]{Sick22}:

\begin{equation}
    Q_{ZV} = 0,2 \frac{kWh}{d \cdot m} \cdot \frac{t_Z}{24} \cdot L_Z
    \label{eq:Zirkulation}
\end{equation}
\begin{center}
    tägliche Betriebszeit $t_Z = 24 h$\\
    Leitungslänge $L_Z = 40 m$
\end{center}
\vspace{\baselineskip}
Aus \autoref{eq:Zirkulation} geht ein Zirkulationsbedarf von $Q_{ZV} = 2920 \frac{kWh}{a}$ hervor.\\
Die Summe der Bedarfe ergibt den Gesamtwärmebedarf inklusive aufsummierter Toleranz.
Dieser beläuft sich auf $Q_{Ges}=13631,025 \pm 1643,292 \frac{kWh}{a}$.\\