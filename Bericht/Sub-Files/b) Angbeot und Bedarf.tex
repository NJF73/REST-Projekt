\subsection{b}
Bestimmen Sie den (mittleren) Gesamtwärmebedarf Q des Gebäudes (Warmwasser, Heizung, Zirkulation).
Das Gebäude weist einen unter genormten Randbedingungen berechneten spezifischen jährlichen Heiz-
wärmebedarf von 50 kWh/(m2a) auf. Im Gebäude werden insgesamt 40 m Zirkulationsleitungen verlegt.
Die Zirkulation ist 24 Stunden am Tag eingeschaltet.
Welche Faktoren beeinflussen den tatsächlichen Gesamtwärmeverbrauch? Versuchen Sie auch hier die
Schwankung des Verbrauchs um den mittleren Gesamtwärmebedarf
 %𝑄 ± ∆𝑄 
 qualifiziert zu schätzen
(Begründung!)